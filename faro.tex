%Requiring precision assembly in tough industrial environments, components used
% in shipbuilding and naval engineering must be verified against design specifications to ensure proper fit.

%Providing a fast and accurate way to inspect ship parts and components during
% boat construction, portable 3D measurement solutions from FARO, such as the
% FARO Laser Tracker, can be used to install components directly to the ship
% itself. Digitizing hulls and existing conditions, 3D laser scanning solutions
% from FARO allows for the captures of 3D as-built documentation for repair or
% restoration.

A \textregistered{FARO} possui uma solução comercial de laser 3D portáteis,
como o FARO Laser Tracker, capazes de digitalizar o casco do navio. A solução,
disponível em
\href{http://www.faro.com/measurement-solutions/industries/shipbuilding/2015/02/18/scanning-at-depth-three-dimensional-measurement-of-an-ocean-giant}{FARO
NAVAL}, já foi contratada para a construção do navio QUEEN ELIZABETH, 2014.

O mesmo fornecedor também utilizou um sistema análago, FARO Laser
Tracker ION, no alinhamento das seções do acelerados de partículas ESS BIlBAO na
Espanhas.
\href{http://www.faro.com/measurement-solutions/applications/alignment/2013/02/15/faro-laser-tracker-integrated-into-the-ess-bilbao-particle-accelerator}{FARO
 ACELERADOR}. Outra aplicação do mesmo sistema foi utilizado pela Siemens Metal
 Technologies para o alinhamento de bobinas em sua fábrica de alumínio, onde
 cada carretel precisa estar exatamente a $90^o$ em relação as folhas de
 alumínio.
\href{http://www.faro.com/measurement-solutions/applications/alignment/2012/06/19/perfect-alignment-for-steel-and-aluminium-mills-at-siemens-with-faro-devices}{FARO
SIEMENS}. Finalmente, o projeto ALMA, maior observatório astronomico a radio do
mundo, utilizou o FARO Laser Tracker X para realizar o alinhamento das 16
petálas que formavam cada antena. 
\href{http://www.faro.com/measurement-solutions/applications/alignment/2012/03/02/world-s-largest-radio-astronomical-project-in-the-atacama-desert-in-chile}{FARO
ALMA}

A Nikon Metrology também é um forte fornecedor de sensores a Laser e em
parceria com a Embraer e o ITA implementou um projeto de alinhamento e
nivelamento automático da fuselagem de aviões. O sistema consiste na utilização
de robôs, auxiliados por um sistema de iGPS, que fornece a localização de cada
peça em um ambiente coberto e um Lidar MV 330.
\href{http://blog.nikonmetrology.com/blog/nikon-metrology-supplies-igps-and-laser-radar/}{EMBRAER
NIKON}

A empresa Tecnhnipipe utilizou um scanner a laser para realizar um estudo
preparatório para a instalação de um guindaste em um canteiro de obras do centro
da cidade XXXX. O estudo tinha como objetivo identificar o comprimento e altura
máxima, assim com a rotação permitida pelo guindaste, de maneira que não
houvesse colisão com os prédios e árvores e postes de eletricidade ao redor.
Para a realização dos cálculos foi necessário o escaneamento 3D do local e assim
todas as medidas de distância puderam ser extraídas com facilidade.
\href{http://www.faro.com/products/3d-surveying/laser-scanner-faro-focus-3d/case-studies/2011/11/23/technipipe-diversifies-its-activities-using-a-faro-laser-scanner}{FARO
CRANE}
